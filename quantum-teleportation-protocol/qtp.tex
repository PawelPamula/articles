% ***********************************************************
% ******************* PHYSICS HEADER ************************
% ***********************************************************
% Version 2
\documentclass[10pt]{article}
\usepackage{amsmath}
\usepackage{amsthm} % Theorem Formatting
\usepackage{amssymb}	% Math symbols such as \mathbb
\usepackage{graphicx} % Allows for eps images
\usepackage[dvips,letterpaper,margin=0.75in,bottom=0.5in]{geometry} % Sets margins and page size
\usepackage{multicol}
%\makeatletter % Need for anything that contains an @ command 
%\renewcommand{\maketitle} % Redefine maketitle to conserve space
%{ \begingroup \vskip 10pt \begin{center} \large {\bf \@title}
%	\vskip 10pt \large \@author \hskip 20pt \@date \end{center}
% \vskip 10pt \endgroup \setcounter{footnote}{0} }
\makeatother % End of region containing @ commands
\renewcommand{\labelenumi}{(\alph{enumi})} % Use letters for enumerate
% \DeclareMathOperator{\Sample}{Sample}
\let\vaccent=\v % rename builtin command \v{} to \vaccent{}
\renewcommand{\v}[1]{\ensuremath{\mathbf{#1}}} % for vectors
\newcommand{\gv}[1]{\ensuremath{\mbox{\boldmath$ #1 $}}} 
% for vectors of Greek letters
\newcommand{\uv}[1]{\ensuremath{\mathbf{\hat{#1}}}} % for unit vector
\newcommand{\abs}[1]{\left| #1 \right|} % for absolute value
\newcommand{\avg}[1]{\left< #1 \right>} % for average
\let\underdot=\d % rename builtin command \d{} to \underdot{}
\renewcommand{\d}[2]{\frac{d #1}{d #2}} % for derivatives
\newcommand{\dd}[2]{\frac{d^2 #1}{d #2^2}} % for double derivatives
\newcommand{\pd}[2]{\frac{\partial #1}{\partial #2}} 
% for partial derivatives
\newcommand{\pdd}[2]{\frac{\partial^2 #1}{\partial #2^2}} 
% for double partial derivatives
\newcommand{\pdc}[3]{\left( \frac{\partial #1}{\partial #2}
 \right)_{#3}} % for thermodynamic partial derivatives
%\newcommand{\ket}[1]{\left| #1 \right>} % for Dirac bras
%\newcommand{\bra}[1]{\left< #1 \right|} % for Dirac kets
\newcommand{\braket}[2]{\left< #1 \vphantom{#2} \right|
 \left. #2 \vphantom{#1} \right>} % for Dirac brackets
\newcommand{\matrixel}[3]{\left< #1 \vphantom{#2#3} \right|
 #2 \left| #3 \vphantom{#1#2} \right>} % for Dirac matrix elements
\newcommand{\grad}[1]{\gv{\nabla} #1} % for gradient
\let\divsymb=\div % rename builtin command \div to \divsymb
\renewcommand{\div}[1]{\gv{\nabla} \cdot #1} % for divergence
\newcommand{\curl}[1]{\gv{\nabla} \times #1} % for curl
\let\baraccent=\= % rename builtin command \= to \baraccent
\renewcommand{\=}[1]{\stackrel{#1}{=}} % for putting numbers above =
\newtheorem{prop}{Proposition}
\newtheorem{thm}{Theorem}[section]
\newtheorem{lem}[thm]{Lemma}
\theoremstyle{definition}
\newtheorem{dfn}{Definition}
\theoremstyle{remark}
\newtheorem*{rmk}{Remark}

% ***********************************************************
% ********************** END HEADER *************************
% ***********************************************************
%    Q-circuit version 2
%    Copyright (C) 2004  Steve Flammia & Bryan Eastin
%    Last modified on: 9/16/2011
%
%    This program is free software; you can redistribute it and/or modify
%    it under the terms of the GNU General Public License as published by
%    the Free Software Foundation; either version 2 of the License, or
%    (at your option) any later version.
%
%    This program is distributed in the hope that it will be useful,
%    but WITHOUT ANY WARRANTY; without even the implied warranty of
%    MERCHANTABILITY or FITNESS FOR A PARTICULAR PURPOSE.  See the
%    GNU General Public License for more details.
%
%    You should have received a copy of the GNU General Public License
%    along with this program; if not, write to the Free Software
%    Foundation, Inc., 59 Temple Place, Suite 330, Boston, MA  02111-1307  USA

% Thanks to the Xy-pic guys, Kristoffer H Rose, Ross Moore, and Daniel Müllner,
% for their help in making Qcircuit work with Xy-pic version 3.8.  
% Thanks also to Dave Clader, Andrew Childs, Rafael Possignolo, Tyson Williams,
% Sergio Boixo, Cris Moore, Jonas Anderson, and Stephan Mertens for helping us test 
% and/or develop the new version.

\usepackage{xy}
\xyoption{matrix}
\xyoption{frame}
\xyoption{arrow}
\xyoption{arc}

\usepackage{ifpdf}
\ifpdf
\else
\PackageWarningNoLine{Qcircuit}{Qcircuit is loading in Postscript mode.  The Xy-pic options ps and dvips will be loaded.  If you wish to use other Postscript drivers for Xy-pic, you must modify the code in Qcircuit.tex}
%    The following options load the drivers most commonly required to
%    get proper Postscript output from Xy-pic.  Should these fail to work,
%    try replacing the following two lines with some of the other options
%    given in the Xy-pic reference manual.
\xyoption{ps}
\xyoption{dvips}
\fi

% The following resets Xy-pic matrix alignment to the pre-3.8 default, as
% required by Qcircuit.
\entrymodifiers={!C\entrybox}

\newcommand{\bra}[1]{{\left\langle{#1}\right\vert}}
\newcommand{\ket}[1]{{\left\vert{#1}\right\rangle}}
    % Defines Dirac notation. %7/5/07 added extra braces so that the commands will work in subscripts.
\newcommand{\qw}[1][-1]{\ar @{-} [0,#1]}
    % Defines a wire that connects horizontally.  By default it connects to the object on the left of the current object.
    % WARNING: Wire commands must appear after the gate in any given entry.
\newcommand{\qwx}[1][-1]{\ar @{-} [#1,0]}
    % Defines a wire that connects vertically.  By default it connects to the object above the current object.
    % WARNING: Wire commands must appear after the gate in any given entry.
\newcommand{\cw}[1][-1]{\ar @{=} [0,#1]}
    % Defines a classical wire that connects horizontally.  By default it connects to the object on the left of the current object.
    % WARNING: Wire commands must appear after the gate in any given entry.
\newcommand{\cwx}[1][-1]{\ar @{=} [#1,0]}
    % Defines a classical wire that connects vertically.  By default it connects to the object above the current object.
    % WARNING: Wire commands must appear after the gate in any given entry.
\newcommand{\gate}[1]{*+<.6em>{#1} \POS ="i","i"+UR;"i"+UL **\dir{-};"i"+DL **\dir{-};"i"+DR **\dir{-};"i"+UR **\dir{-},"i" \qw}
    % Boxes the argument, making a gate.
\newcommand{\meter}{*=<1.8em,1.4em>{\xy ="j","j"-<.778em,.322em>;{"j"+<.778em,-.322em> \ellipse ur,_{}},"j"-<0em,.4em>;p+<.5em,.9em> **\dir{-},"j"+<2.2em,2.2em>*{},"j"-<2.2em,2.2em>*{} \endxy} \POS ="i","i"+UR;"i"+UL **\dir{-};"i"+DL **\dir{-};"i"+DR **\dir{-};"i"+UR **\dir{-},"i" \qw}
    % Inserts a measurement meter.
    % In case you're wondering, the constants .778em and .322em specify
    % one quarter of a circle with radius 1.1em.
    % The points added at + and - <2.2em,2.2em> are there to strech the
    % canvas, ensuring that the size is unaffected by erratic spacing issues
    % with the arc.
\newcommand{\measure}[1]{*+[F-:<.9em>]{#1} \qw}
    % Inserts a measurement bubble with user defined text.
\newcommand{\measuretab}[1]{*{\xy*+<.6em>{#1}="e";"e"+UL;"e"+UR **\dir{-};"e"+DR **\dir{-};"e"+DL **\dir{-};"e"+LC-<.5em,0em> **\dir{-};"e"+UL **\dir{-} \endxy} \qw}
    % Inserts a measurement tab with user defined text.
\newcommand{\measureD}[1]{*{\xy*+=<0em,.1em>{#1}="e";"e"+UR+<0em,.25em>;"e"+UL+<-.5em,.25em> **\dir{-};"e"+DL+<-.5em,-.25em> **\dir{-};"e"+DR+<0em,-.25em> **\dir{-};{"e"+UR+<0em,.25em>\ellipse^{}};"e"+C:,+(0,1)*{} \endxy} \qw}
    % Inserts a D-shaped measurement gate with user defined text.
\newcommand{\multimeasure}[2]{*+<1em,.9em>{\hphantom{#2}} \qw \POS[0,0].[#1,0];p !C *{#2},p \drop\frm<.9em>{-}}
    % Draws a multiple qubit measurement bubble starting at the current position and spanning #1 additional gates below.
    % #2 gives the label for the gate.
    % You must use an argument of the same width as #2 in \ghost for the wires to connect properly on the lower lines.
\newcommand{\multimeasureD}[2]{*+<1em,.9em>{\hphantom{#2}} \POS [0,0]="i",[0,0].[#1,0]="e",!C *{#2},"e"+UR-<.8em,0em>;"e"+UL **\dir{-};"e"+DL **\dir{-};"e"+DR+<-.8em,0em> **\dir{-};{"e"+DR+<0em,.8em>\ellipse^{}};"e"+UR+<0em,-.8em> **\dir{-};{"e"+UR-<.8em,0em>\ellipse^{}},"i" \qw}
    % Draws a multiple qubit D-shaped measurement gate starting at the current position and spanning #1 additional gates below.
    % #2 gives the label for the gate.
    % You must use an argument of the same width as #2 in \ghost for the wires to connect properly on the lower lines.
\newcommand{\control}{*!<0em,.025em>-=-<.2em>{\bullet}}
    % Inserts an unconnected control.
\newcommand{\controlo}{*+<.01em>{\xy -<.095em>*\xycircle<.19em>{} \endxy}}
    % Inserts a unconnected control-on-0.
\newcommand{\ctrl}[1]{\control \qwx[#1] \qw}
    % Inserts a control and connects it to the object #1 wires below.
\newcommand{\ctrlo}[1]{\controlo \qwx[#1] \qw}
    % Inserts a control-on-0 and connects it to the object #1 wires below.
\newcommand{\targ}{*+<.02em,.02em>{\xy ="i","i"-<.39em,0em>;"i"+<.39em,0em> **\dir{-}, "i"-<0em,.39em>;"i"+<0em,.39em> **\dir{-},"i"*\xycircle<.4em>{} \endxy} \qw}
    % Inserts a CNOT target.
\newcommand{\qswap}{*=<0em>{\times} \qw}
    % Inserts half a swap gate.
    % Must be connected to the other swap with \qwx.
\newcommand{\multigate}[2]{*+<1em,.9em>{\hphantom{#2}} \POS [0,0]="i",[0,0].[#1,0]="e",!C *{#2},"e"+UR;"e"+UL **\dir{-};"e"+DL **\dir{-};"e"+DR **\dir{-};"e"+UR **\dir{-},"i" \qw}
    % Draws a multiple qubit gate starting at the current position and spanning #1 additional gates below.
    % #2 gives the label for the gate.
    % You must use an argument of the same width as #2 in \ghost for the wires to connect properly on the lower lines.
\newcommand{\ghost}[1]{*+<1em,.9em>{\hphantom{#1}} \qw}
    % Leaves space for \multigate on wires other than the one on which \multigate appears.  Without this command wires will cross your gate.
    % #1 should match the second argument in the corresponding \multigate.
\newcommand{\push}[1]{*{#1}}
    % Inserts #1, overriding the default that causes entries to have zero size.  This command takes the place of a gate.
    % Like a gate, it must precede any wire commands.
    % \push is useful for forcing columns apart.
    % NOTE: It might be useful to know that a gate is about 1.3 times the height of its contents.  I.e. \gate{M} is 1.3em tall.
    % WARNING: \push must appear before any wire commands and may not appear in an entry with a gate or label.
\newcommand{\gategroup}[6]{\POS"#1,#2"."#3,#2"."#1,#4"."#3,#4"!C*+<#5>\frm{#6}}
    % Constructs a box or bracket enclosing the square block spanning rows #1-#3 and columns=#2-#4.
    % The block is given a margin #5/2, so #5 should be a valid length.
    % #6 can take the following arguments -- or . or _\} or ^\} or \{ or \} or _) or ^) or ( or ) where the first two options yield dashed and
    % dotted boxes respectively, and the last eight options yield bottom, top, left, and right braces of the curly or normal variety.  See the Xy-pic reference manual for more options.
    % \gategroup can appear at the end of any gate entry, but it's good form to pick either the last entry or one of the corner gates.
    % BUG: \gategroup uses the four corner gates to determine the size of the bounding box.  Other gates may stick out of that box.  See \prop.

\newcommand{\rstick}[1]{*!L!<-.5em,0em>=<0em>{#1}}
    % Centers the left side of #1 in the cell.  Intended for lining up wire labels.  Note that non-gates have default size zero.
\newcommand{\lstick}[1]{*!R!<.5em,0em>=<0em>{#1}}
    % Centers the right side of #1 in the cell.  Intended for lining up wire labels.  Note that non-gates have default size zero.
\newcommand{\ustick}[1]{*!D!<0em,-.5em>=<0em>{#1}}
    % Centers the bottom of #1 in the cell.  Intended for lining up wire labels.  Note that non-gates have default size zero.
\newcommand{\dstick}[1]{*!U!<0em,.5em>=<0em>{#1}}
    % Centers the top of #1 in the cell.  Intended for lining up wire labels.  Note that non-gates have default size zero.
\newcommand{\Qcircuit}{\xymatrix @*=<0em>}
    % Defines \Qcircuit as an \xymatrix with entries of default size 0em.
\newcommand{\link}[2]{\ar @{-} [#1,#2]}
    % Draws a wire or connecting line to the element #1 rows down and #2 columns forward.
\newcommand{\pureghost}[1]{*+<1em,.9em>{\hphantom{#1}}}
    % Same as \ghost except it omits the wire leading to the left. 

\usepackage[utf8]{inputenc}
\usepackage{datetime}
\usepackage{multicol}
\usepackage{enumitem}

\newdateformat{mydate}{\monthname[\THEMONTH] \THEDAY, \THEYEAR}

\title{Quantum Teleportation Protocol}
\author{Paweł Pamuła}
\date{\mydate\today}
\begin{document}
\maketitle

%%%%%%%%%%%%%%%%%%%%%%%%%%%%%%%%%%%%%%%%%%%%%
\begin{abstract}
A brief description of relatively simple quantum teleportation protocol, which allows transmiting a qubit from one place to another, without physically transferring it through the space.
\end{abstract}
\vfill
\begin{multicols*}{2}
\section{Qubit}
\indent
Qubit is a linear superposition of two arbitralily chosen orthonormal basis vectors. 
\begin{align*}
\ket{\psi} = \alpha_0\ket{0} + \alpha_1\ket{1}
\end{align*}
Unit vector $\ket{0}$ is $[1, 0]^T$ and $\ket{1}$ is $[0,1]^T$. $\alpha_0$ and $\alpha_1$ denote \textit{probability amplitudes} and they are complex numbers in general. Probability of outcome $\ket{0}$ is $\abs{\alpha_0}^2$ and $\ket{1}$ is $\abs{\alpha_1}^2$. $\ket{\psi}$ is an unit vector from two-dimensional Hilber space $\mathcal{H}$, therefore $\abs{\alpha_0}^2 + \abs{\alpha_1}^2 = 1$. It is consistent with our empirical intuiton - after measurment one obtains a well-defined state $\ket{0}$ or $\ket{1}$ and there is no other possibility.

To measure a state we have to choose an orthonormal basis and probability of the outcome is equal to the square of the length of a projection onto basis vector. After measurement state collapses to the basis vector.

We can perform measurments in arbitralily chosen basis. For instance, \textit{sign basis} is defined as
\begin{align*}
\ket{+} = \frac{1}{\sqrt{2}}(\ket{0} + \ket{1})\\
\ket{-} = \frac{1}{\sqrt{2}}(\ket{0} - \ket{1})
\end{align*}
and $\ket{\psi}$ can be easilly rewritten in this basis as
\begin{align*}
\ket{\psi} = \frac{\alpha_0 + \alpha_1}{\sqrt{2}}\ket{+}+\frac{\alpha_0-\alpha_1}{\sqrt{2}}\ket{-}
\end{align*}
%%%%%%%%%%%%%%%%%%%%%%%%%%%%%%%%%%%%%%%%%%%%%
\section{Composite systems}
State of composite system consisting of two qubits is described as a tensor product of two states. Suppose the first qubit is in state $\ket{\psi_1} = \alpha_0\ket{0} + \alpha_1\ket{1}$ and second is $\ket{\psi_2} = \beta_0\ket{0} + \beta_1\ket{1}$. We can interprete composite state $\ket{\psi}$ which is equal to $\ket{\psi_1}\otimes\ket{\psi_2}$ as $\ket{\psi_1\psi_2}$.
\begin{align}
\begin{split}
\label{eq:composite}
\ket{\psi} &=  (\alpha_0\ket{0} + \alpha_1\ket{1})\otimes(\beta_0\ket{0} + \beta_1\ket{1}) \\
&=\alpha_0\beta_0\ket{00} + \alpha_0\beta_1\ket{01} + \alpha_1\beta_0\ket{10} + \alpha_1\beta_1\ket{11}
\end{split}
\end{align}

Given two independent qubits, we can always obtain state of such composite system. But it is not always possible to obtain tensor product from two-qubit state. Suppose there is state $\ket{\psi} = \frac{1}{\sqrt{2}}\ket{00}+\frac{1}{\sqrt{2}}\ket{11}$. We would like to rewrite as $(\alpha_0\ket{0} + \alpha_1\ket{1})\otimes(\beta_0\ket{0} + \beta_1\ket{1})$. According to \eqref{eq:composite}, $\alpha_0\beta_0 = \frac{1}{\sqrt{2}}$ and $\alpha_1\beta_1 = \frac{1}{\sqrt{2}}$, therefore $\alpha_0, \alpha_1, \beta_0, \beta_1 \neq 0$. On the other hand $\alpha_0\beta_1 = 0$, which obviously is a contradiction - we cannot factorize those kind of states called \textit{entangled states}.

We can perform partial measurment on such composite systems. For example, we can measure only the first qubit  and obtain $\ket{0}$ with probability $P[\ket{0}]$ and $\ket{1}$ with probability $P[\ket{1}]$
\begin{align*}
P[\ket{0}] = \abs{\alpha_0\beta_0}^2 + \abs{\alpha_0\beta_1}^2\\
P[\ket{1}] = \abs{\alpha_1\beta_0}^2 + \abs{\alpha_1\beta_1}^2
\end{align*}
After measurment there is a new state.
\begin{align*}
\ket{\boldsymbol{0}\psi'} = \frac{1}{\sqrt{ \abs{\alpha_0\beta_0}^2 + \abs{\alpha_0\beta_1}^2}}\left(\alpha_0\beta_0\ket{\boldsymbol{0}0} + \alpha_0\alpha_1\ket{\boldsymbol{0}1}\right)
\end{align*}

There are four states which represent the simplest examples of entanglement of two qubits called \textit{Bell states}.
\begin{align*}
\begin{split}
\label{eq:bell_states}
\ket{\psi^+} &=  \frac{1}{\sqrt{2}}(\ket{00}+\ket{11})\\
\ket{\psi^-} &=  \frac{1}{\sqrt{2}}(\ket{00}-\ket{11})\\
\ket{\phi^+} &=  \frac{1}{\sqrt{2}}(\ket{01}+\ket{10})\\
\ket{\phi^-} &=  \frac{1}{\sqrt{2}}(\ket{01}-\ket{10})
\end{split}
\end{align*}

Bell states have this interesting property - two qubits that are in entangled state can be spatially separated but they are not subject to relativistic limitations. When someone measures the first qubit there are two possible outcomes. In each case, any subsequent measurement in the same basis will always return the same result with certainity. 
%%%%%%%%%%%%%%%%%%%%%%%%%%%%%%%%%%%%%%%%%%%%%
\section{Evolution of  a qubit}
Evolution of qubit can be described as a rotation of an unit vector in two-dimensional Hilbert space. Transformation can be represented by operators applied to a vector. Those operators are unitary, therefore they preserve the angles between vectors and their lengths. We can think of those transformation as quantum gates. A gate which acts $k$ one qubit is represented by $2^k\times 2^k$ matrix. There are three basic one-qubit gate necessary to describe quantum teleportation protocol.
\[
X = 
\begin{bmatrix}
0 & 1\\
1 & 0
\end{bmatrix}
Z = 
\begin{bmatrix}
1 & 0\\
0 & -1
\end{bmatrix}
H = \frac{1}{\sqrt{2}} 
\begin{bmatrix}
1 & 1\\
1 & -1
\end{bmatrix}
\]
X is equal to a rotation around X-axis by $\pi$ radians.
\begin{align*}
X(\alpha_0\ket{0} + \alpha_1\ket{1}) &=
\begin{bmatrix}
0 & 1\\
1 & 0
\end{bmatrix}
\cdot
\begin{bmatrix}
\alpha_0\\
\alpha_1
\end{bmatrix}
=
\begin{bmatrix}
\alpha_0\\
\alpha_1
\end{bmatrix} = \\
&= \alpha_1\ket{0} + \alpha_0\ket{1}
\end{align*}
Z is called phase flip and it is equal to a rotation around Z-axis by $\pi$ radians.
\begin{align*}
Z(\alpha_0\ket{0} + \alpha_1\ket{1}) &=
\begin{bmatrix}
1 & 0\\
0 & -1
\end{bmatrix}
\cdot
\begin{bmatrix}
\alpha_0\\
\alpha_1
\end{bmatrix}
=
\begin{bmatrix}
\alpha_0\\
-\alpha_1
\end{bmatrix} = \\
&= \alpha_0\ket{0} - \alpha_1\ket{1}
\end{align*}
H is called Hadamard gate and it changes the basis from $\{\ket{0},\ket{1}\}$ to $\{\ket{+},\ket{-}\}$ and the other way around. 
\begin{align*}
H(\alpha_0\ket{0} + \alpha_1\ket{1}) &= \frac{1}{\sqrt{2}}
\begin{bmatrix}
1 & 1\\
1 & -1
\end{bmatrix}
\cdot
\begin{bmatrix}
\alpha_0\\
\alpha_1
\end{bmatrix}\\
&= \frac{1}{\sqrt{2}}
\begin{bmatrix}
\alpha_0+\alpha_1\\
\alpha_0-\alpha_1
\end{bmatrix} = \\
&= \frac{\alpha_0+\alpha_1}{\sqrt{2}}\ket{0} - \frac{\alpha_0-\alpha_1}{\sqrt{2}}\ket{1} 
\end{align*}
One application of the Hadamard gate to either a 0 or 1 qubit will produce a quantum state that, if observed, will be a 0 or 1 with equal probability.

There are gates acting or two cubits, where one of them act as a control for operation. One of them example is CNOT (controlled NOT) gate. 
\begin{align*}
CNOT = 
\begin{bmatrix}
1 & 0 & 0 & 0\\
0 & 1 & 0 & 0\\
0 & 0 & 0 & 1\\
0 & 0 & 1 & 0
\end{bmatrix}
\end{align*}
\begin{align*}
\Qcircuit @C=1em @R=1em {
\lstick{a} & \ctrl{1} & \rstick{a} \qw \\
\lstick{b} & \targ & \rstick{a\oplus b} \qw
}
\end{align*}
This particular CNOT gate takes the first bit as a control bit and if it is equal to 1 it flips the second one. 

We can combine those gates in more complicated circuits. Let's consider circuit involving Hadamard and CNOT gate and input state $\ket{00}$
\begin{align*}
\Qcircuit @C=1em @R=1em {
\lstick{\ket{0}} & \gate{H} & \qw & \ctrl{1} & \qw\\
\lstick{\ket{0}} & \qw & \qw & \targ & \qw
}
\end{align*}
It maps the first bit to $\ket{+}$ and performs control not using first bit as control. 
\begin{align*}
\ket{00}\longrightarrow\underbrace{\frac{1}{\sqrt{2}}(\ket{0} + \ket{1})}_{\ket{+}}\ket{0} = \frac{1}{\sqrt{2}}(\ket{00}+\ket{11})
\end{align*}
The result is a previously mentioned Bell state. by varying the input we can obtain other three Bell states, which proves, that using basic quantum gates we are able to produce entangled states.

\section{No-cloning theorem}
Suppose our task is to construc a quantum circuit that accepts 2 qubits at the input - one is unknown superposition of 2 states ($\alpha_0$ and $\alpha_1$ are unknown, obtained as a result of an experiment for instance) and $\ket{s}$ is a well-known state, such as $\ket{0}$. 

\begin{align*}
\Qcircuit @C=1em @R=1em {
\lstick{\ket{\psi}} & \multigate{1}{U} & \rstick{\ket{\psi}} \qw \\
\lstick{\ket{s}} & \ghost{U} & \rstick{\ket{\psi}} \qw
}
\end{align*}

When $\ket{\psi}= \alpha_0\ket{0}+\alpha_1\ket{1}$, by linearity of unitary operator $U$ we obtain:
\begin{align*}
U[(\alpha_0\ket{0}+\alpha_1\ket{1})\ket{s}] = \alpha_0\underbrace{U\ket{0}\ket{s}}_{\ket{00}} + \alpha_1 \underbrace{U\ket{1}\ket{s}}_{\ket{11}}
\end{align*}
$U$ is construct in a such way, that $U\ket{\psi}\ket{s} = \ket{\psi}\ket{\psi}$. On the other hand output is a tensor product of two states $\ket{\psi}$.
\begin{align*}
\ket{\psi\psi} &= (\alpha_0\ket{0}+\alpha_1\ket{1})\otimes(\alpha_0\ket{0}+\alpha_1\ket{1})\\
&= \alpha_0^2\ket{00}+\alpha_0\alpha_1\ket{01}+\alpha_1\alpha_0\ket{10}+\alpha_1^2\ket{11}
\end{align*}
which obviously is a contradiction. We conclude, that there is no such quantum circuit, that can clone an unknown quantum state.

\section{Quantum teleportation}
Quantum teleportation protocol is based on basic one and two-qubit gates. Prerequisites for this protocol are an unknown qubit, entangled state shared between sender and receiver and a set of quantum gates. Assume Alice (sender) has an uknown qubit - coefficients $\alpha_0, \alpha_1$ are unknown. She would like Bob to have this state, for example to continue an experiment using some devices located in another place. 

\begin{align*}
\Qcircuit @C=1em @R=2em {
\lstick{\ket{\psi}} & & \ctrl{1} & \qw & \qw & \gate{H} & \meter & \control\cw & & \\
\lstick{\ket{\phi_A}} & & \targ & \meter & \control\cw & & & \cwx &\\ 
\lstick{\ket{\phi_B}} & & \qw & \qw & \gate{H}\cwx &\qw & \qw & \gate{Z}\cwx\qw & \rstick{\ket{\psi}}\qw
\gategroup{1}{3}{2}{4}{1.6em}{.}
\gategroup{1}{6}{3}{8}{.6em}{.}
}
\end{align*}

 Let's consider first group of gates. Alice has\\ $\ket{\psi} = \alpha_0\ket{0} + \alpha_1\ket{1}$ and one of two qubits $\ket{\phi_A}$ in entagled state.
\begin{enumerate}[leftmargin=0cm,itemindent=.5cm,labelwidth=\itemindent,labelsep=0cm,align=left]
\item[1)]In the first step, Alice performs CNOT operation with first bit as a control bit and second one as a result bit.
\begin{align*}
CNOT\ket{\psi}\ket{\phi_A} &= \frac{\alpha_0}{\sqrt{2}}(\ket{000} + \ket{011}) + \frac{\alpha_1}{\sqrt{2}}(\ket{100}+\ket{111})
\end{align*}
Following result:
\begin{align*}
\ket{\psi}\ket{\phi_A} &= \frac{\alpha_0}{\sqrt{2}}(\ket{000} + \ket{011}) + \frac{\alpha_1}{\sqrt{2}}(\ket{110}+\ket{101})
\end{align*}

\item[2)] Alice measures the second bit in the composite system. There are two possible results
\begin{align*}
0\longrightarrow(\alpha_0\ket{00}+\alpha_1\ket{11})\\
1\longrightarrow(\alpha_0\ket{01}+\alpha_1\ket{10})
\end{align*}
In the second case Alice contacts Bob through classical medium and tells him to apply bit flip (X) gate.
\item[3)] Alice applies Hadamard gate to the first bit in the composite system.
\begin{align*}
&\frac{\alpha_0}{\sqrt{2}}(\ket{+}+\ket{-})\ket{0}+\frac{\alpha_1}{\sqrt{2}}(\ket{+}-\ket{-})\ket{1} = \\
&\frac{1}{\sqrt{2}}\ket{+}(\alpha_0\ket{0}+\alpha_1\ket{1})+\frac{1}{\sqrt{2}}\ket{-}(\alpha_0\ket{0}-\alpha_1\ket{1})
\end{align*}
\item[4)] Alice performs measurement in sign basis. If the results is $\ket{+}$ entangled state is set to initial $\ket{\psi}$. In other case she contacts Bob, who should apply phase flip (Z) gate. It turns out that Bob has initial unknown state $\ket{\psi}$.
\end{enumerate}
\section{Summary}
First of all, this protocol does not violate no-cloning theorem, because Alice measures both her bits and after measurment they are destroyed. Information also does not violate theory of relativity, because Bob has to receive message via classical medium how to apply X and Z gate. As the result Bob receives initial unknown states $\ket{\psi}$.
\end{multicols*}
\end{document}